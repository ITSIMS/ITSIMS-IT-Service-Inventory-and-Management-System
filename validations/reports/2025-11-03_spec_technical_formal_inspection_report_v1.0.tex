\documentclass[12pt, a4paper]{article}

% --- ПРЕАМБУЛА ---
\usepackage[T2A]{fontenc}
\usepackage[utf8]{inputenc}
\usepackage[russian]{babel}
\usepackage[left=2cm, right=1.5cm, top=2cm, bottom=2cm]{geometry}
\usepackage{tabularx}
\usepackage{pdflscape} % Для альбомной ориентации таблицы
\usepackage{longtable} % Для таблиц на несколько страниц
\usepackage{array}     % Для улучшения контроля над колонками

% --- ОТКЛЮЧЕНИЕ НУМЕРАЦИИ СТРАНИЦ ---
\pagestyle{empty}

% --- НАЧАЛО ДОКУМЕНТА ---
\begin{document}

\centering
{\Huge\bfseries Отчёт о формальной инспекции ТЗ\par}
\vspace{2cm}

\begin{flushleft}
\large
\textbf{Версия ТЗ:} 1 \\
\vspace{0.5cm}
\textbf{Дата инспекции:} 01.11.2025 \\
\end{flushleft}
\vspace{1.5cm}
\hrule
\vspace{1.5cm}

% --- РАЗДЕЛЫ БЕЗ НУМЕРАЦИИ (используем \section*{...}) ---

\section*{Участники процесса}
\begin{tabularx}{\textwidth}{|l|X|}
    \hline
    \textbf{Роль} & \textbf{ФИО} \\
    \hline
    \textbf{Автор документа} & Зацепилин А.В. \\
    \hline
    \textbf{Инспектор} & Зацепилин А.В. \\
    \hline
    \textbf{Эксперт} & Зацепилин А.В. \\
    \hline
    \textbf{Модератор} & Зацепилин А.В. \\
    \hline
\end{tabularx}
\vspace{1cm}

\section*{Идентификация и проверка формальных свойств объекта инспекции}
\begin{flushleft}
\large
\noindent\textbf{Объект инспекции (ссылка на репозиторий):} https://github.com/ITSIMS/ITSIMS-IT-Service-Inventory-and-Management-System \\
\vspace{0.5cm}
\noindent\textbf{Версия объекта (хэш коммита / тег):} 2b8d06c \\
\vspace{0.5cm}
\noindent\textbf{Состояние объекта в жизненном цикле (готов к инспекции):} Готов к инспекции (статус "Согласован с заказчиком") \\
\end{flushleft}
\vspace{1cm}


\begin{landscape}
\footnotesize
\begin{longtable}{|p{0.5cm}|p{4.5cm}|p{7cm}|p{1.5cm}|p{11.5cm}|}
\hline
\textbf{№} & \textbf{Критерий проверки} & \textbf{Описание} & \textbf{Статус (+/-)} & \textbf{Замечания} \\
\hline
\endhead

% --- ЧАСТЬ I: ФОРМАЛЬНАЯ ПРОВЕРКА И СТРУКТУРА ---
\multicolumn{5}{|l|}{\textbf{Часть I: Формальная проверка и структура документа}} \\
\hline
1 & \textbf{Титульный лист и метаданные} & Присутствуют название проекта, автор, дата, версия документа и история изменений. & + & Все необходимые метаданные (версия, дата, автор, история изменений) присутствуют в начале документа. \\
\hline
2 & \textbf{Структура и форматирование} & Документ соответствует корпоративному шаблону. Разделы и пункты пронумерованы корректно. Все рисунки и таблицы имеют подписи и номера. & + & Документ имеет чёткую и логичную структуру. Нумерация разделов корректна. Форматирование соответствует стандарту Markdown. \\
\hline
3 & \textbf{Терминология и грамотность} & Присутствует глоссарий. Термины используются единообразно. Отсутствуют орфографические и грамматические ошибки. & + & Присутствует раздел 4 "Термины и определения". Терминология в документе используется последовательно. \\
\hline
4 & \textbf{Доступность документа} & Документ размещён в общем репозитории, ссылка на него актуальна и доступна для всех участников. & + & Документ размещён в публичном репозитории GitHub, ссылка актуальна и доступна. \\
\hline

% --- ЧАСТЬ II: ПОЛНОТА И СОДЕРЖАНИЕ ТРЕБОВАНИЙ ---
\multicolumn{5}{|l|}{\textbf{Часть II: Полнота и содержание требований}} \\
\hline
5 & \textbf{Цели и границы проекта} & Четко описаны бизнес-цели, назначение системы, а также то, что входит и НЕ входит в рамки проекта. & +/- & Цели и назначение системы (раздел 2) описаны чётко. Однако в документе отсутствуют явные границы проекта (что НЕ входит в рамки разработки). \\
\hline
6 & \textbf{Полнота функциональных требований} & Описаны все ключевые функции, пользовательские сценарии и/или пользовательские истории. & + & Раздел 5.1 достаточно полно описывает ключевые функции системы: управление реестром, иерархию, поиск, логирование и интеграцию. \\
\hline
7 & \textbf{Полнота нефункциональных требований} & Заданы требования к производительности, надёжности, масштабируемости, удобству использования и т.д. & + & Присутствуют базовые нефункциональные требования к надёжности, доступности и производительности (раздел 5.3). \\
\hline
8 & \textbf{Требования к интерфейсам и интеграции} & Определены все внешние и внутренние API, протоколы взаимодействия и форматы данных для интеграции с другими системами. & + & Требования к интеграции (раздел 5.2) хорошо проработаны, указаны API, стандарты и протоколы, а также интеграция с LDAP/AD. \\
\hline
9 & \textbf{Требования к безопасности} & Описаны требования по аутентификации, авторизации, шифрованию данных и защите от известных уязвимостей. & + & Описаны основные требования к безопасности (раздел 5.4), включая аутентификацию, ролевую модель и журналирование. \\
\hline
10 & \textbf{Обработка ошибок и исключительных ситуаций} & Определено поведение системы в случае сбоев, неверного ввода данных и других нештатных ситуаций. & - & В ТЗ полностью отсутствует раздел, описывающий требования к обработке ошибок и поведению системы в нештатных ситуациях. \\
\hline

% --- ЧАСТЬ III: КАЧЕСТВО И АТРИБУТЫ ТРЕБОВАНИЙ ---
\multicolumn{5}{|l|}{\textbf{Часть III: Качество и атрибуты требований}} \\
\hline
11 & \textbf{Атомарность и однозначность} & Каждое требование описывает одну и только одну вещь, не допуская двойного толкования или расплывчатых формулировок. & + & Требования в большинстве своем сформулированы однозначно и атомарно, что упрощает их понимание и реализацию. \\
\hline
12 & \textbf{Проверяемость и измеримость} & Для каждого требования можно сформулировать чёткий критерий приёмки и разработать тест-кейс. Формулировки измеримы (например, "время отклика не более 2 секунд"). & + & Требования измеримы. \\
\hline
13 & \textbf{Прослеживаемость до бизнес-целей} & Каждое требование может быть отслежено до бизнес-цели или запроса от заинтересованной стороны. Понятно, ЗАЧЕМ оно нужно. & + & Функциональные требования хорошо прослеживаются до целей, описанных в разделе 2 "Назначение и цели создания системы". \\
\hline
14 & \textbf{Приоритизация требований} & Требованиям присвоены приоритеты, что помогает планировать итерации разработки. & - & В документе полностью отсутствует приоритизация требований. Это является критичным недостатком для планирования разработки. \\
\hline
\end{longtable}
\end{landscape}

\section*{Результаты формальной инспекции}

\noindent\textbf{Итоговая оценка:} Принять с доработками.

\vspace{1cm}
\noindent\textbf{Рекомендации по исправлению:}
\begin{itemize}
    \item Добавить раздел "Границы проекта", где будет явно указано, что не входит в рамки проекта, чтобы избежать разночтений в будущем.
    \item Добавить раздел "Требования к обработке ошибок и исключительных ситуаций", описывающий поведение системы при неверном вводе данных и сбоях.
    \item Провести приоритизацию всех требований. Это критически важно для планирования итераций разработки и управления ожиданиями заказчика.
\end{itemize}

\vspace{2cm} % Отступ перед подписями
\hrulefill

\section*{Подписи}
\vspace{1.5cm}
\begin{tabular}{p{5cm} p{5cm}}
    \textbf{Инспектор} & \hrulefill \\
    & \small{(Подпись, ФИО)} \\
    \textbf{Дата:} \hrulefill & \\
\end{tabular}
\vspace{1.5cm}

\begin{tabular}{p{5cm} p{5cm}}
    \textbf{Эксперт} & \hrulefill \\
    & \small{(Подпись, ФИО)} \\
    \textbf{Дата:} \hrulefill & \\
\end{tabular}
\vspace{1.5cm}

\begin{tabular}{p{5cm} p{5cm}}
    \textbf{Автор документа} & \hrulefill \\
    & \small{(Подпись, ФИО)} \\
    \textbf{Дата:} \hrulefill & \\
\end{tabular}

\end{document}